
{\actuality} 
Мотор-колеса являются конкурентоспособным аналогом классической трансмиссии, так как они помогают избежать использования дополнительных передаточных механизмов, освобождают место в подкапотном пространстве, снижают и равномерно распределяют вес транспортного средства. 
При этом, одной из особенностей использования мотор-колес является необходимость отказа от коробки передач. В этом случае подход к разработке тяговой установки с использованием мотор-колес значительно ограничен в выборе силовых агрегатов: механическая характеристика мотора должна позволять транспортному средству работать в широком диапазоне скоростей без потери крутящего момента на валу. Также существует проблема разработки оптимального алгоритма рекуперации одновременно с четырех источников энергии. Все вышеперечисленные проблемы затрудняют внедрение мотор-колес как основных тяговых агрегатов, поэтому задача разработки топологии и алгоритма управления электромотором в составе мотор-колес, которые позволят обеспечить эффективность работы электромоторов за счет применения системы рекуперации и распределения электроэнергии с увеличением диапазона рабочих скоростей является актуальной.
В работе \cite{Casadei} была представлена топология двойного двухуровнего инвертора с двумя различными источниками энергии на обоих концах. Были рассмотрены алгоритмы распределения заряда между батареями, выведена зависимость коэффициента распределения энергии инверторов от коэффициента модуляции. В \cite{Sun}, а также в \cite{leey}, была рассмотрена возможность компенсации противо-ЭДС  синхронного двигателя с постоянными магнитами  в режимах ослабления поля за счет подключения к концам обмоток двухуровнего инвертера с компенсирующей емкостью. В \cite{Loncarski} было проведено сравнение параметров электромотора, подключенного к двум батареям с эквивалентной емкостью к началу и концу обмоток с электромотором, подключенным с одной стороны к батарее с двукратной емкостью с одной стороны обмоток (концы обмоток подключены по схеме «звезда»). Исследования показали, что, при эквивалентной емкости, электромотор к подключенными с обоих концов инверторами имеет меньшие показателями пульсации тока, а, следовательно, и более высокий КПД. Также данная система имеет возможность резервирования (шунтирование концов обмоток инвертора при неисправности одной из батарей), что повышает надежность системы. В работе \cite{Attaianese} была  рассмотрена совместная работа двух электромоторов, начало обмоток которых было подключено инверторам с тяговыми батареями разной емкости, а их концы к инвертору с конденсаторной батареей. В \cite{Cordopatri} был рассмотрен вопрос выбора оптимального выбора мощности мотора и емкости тяговых батарей для передней и задней оси электромотора. В ходе исследований было выявлено, что наиболее дешевым и более выгодным в весе является конфигурация с установкой на одну из осей моторов с малой скоростью и высокими показателями крутящего момента, а на другую высокоскоростных моторов с более низкими моментными характеристиками. При этом каждая пара моторов получает энергию от источников с разными параметрами напряжения и емкости.
Однако, работ, в комплексе рассматривающих работу системы из четырех электромоторов, с несколькими тяговыми батареями и конденсатором, работающим в режиме рекуперации при торможении, а в режиме высоких скоростей работающий как компенсатор противо-ЭДС моторов для достижения высоких рабочих скоростей с возможностью перераспределения зарядов между тяговыми батареями на данный момент не было представлено.


%\ifsynopsis
%Этот абзац появляется только в~автореферате.
%Для формирования блоков, которые будут обрабатываться только в~автореферате,
%заведена проверка условия \verb!\!\verb!ifsynopsis!.
%Значение условия задаётся в~основном файле документа (\verb!synopsis.tex! для
%автореферата).
%\else
%Этот абзац появляется только в~диссертации.
%Через проверку условия \verb!\!\verb!ifsynopsis!, задаваемого в~основном файле
%документа (\verb!dissertation.tex! для диссертации), можно сделать новую
%команду, обеспечивающую появление цитаты в~диссертации, но~не~в~автореферате.
%\fi

% {\progress}
% Этот раздел должен быть отдельным структурным элементом по
% ГОСТ, но он, как правило, включается в описание актуальности
% темы. Нужен он отдельным структурынм элемементом или нет ---
% смотрите другие диссертации вашего совета, скорее всего не нужен.

{\aim} данной работы является повышение эффективности работы электромотора за счет применения системы рекуперации и распределения электроэнергии с увеличением диапазона рабочих скоростей электромоторов в составе транспортного средства с использованием мотор-колес

Для~достижения поставленной цели необходимо было решить следующие {\tasks}:
\begin{enumerate}
  \item Анализ возможного увеличения диапазона рабочих скоростей электромотора за счет подключения второго инвертора к концам обмоток электромотора с компенсирующей 	емкостью
  \item Анализ эффективности распределения электроэнергии между тяговыми батареями
  \item Анализ эффективности   объединенного инвертора с несколькими электромоторами в 	различных режимах работы.
  \item Разработка алгоритма распределения нагрузки на тяговые колеса с учетом фактической нагруженности электромобиля, а также внешних факторов.
  \item Разработка системы перераспределения энергии между тяговыми батареями для передней и задней оси электромотора посредством конденсаторной батареи
  \item	Математическое моделирование переходных процессов разработанной топологии
  \item Оценка эффективности предлагаемой системы в сравнении с существующими топологиями.
\end{enumerate}


{\novelty}
\begin{enumerate}
  \item Предложена система распределенного электропривода с использованием попарно четырех  СДПМ с разной мощности подключённых к тяговым батареям через ПЧ со стороны начал 	обмоток и общей конденсаторной батареи через ПЧ к концам обмоток
  \item Обоснован способ управления системой распределения заряда емкостей и тяговых батарей, заключающийся в использовании конденсаторной батареи в качестве накопителя энергии во время торможения, а также использовании ее как звена распределения заряда между тяговыми батареями
  \item Разработан способ определения оптимальных параметров мощности электромоторов для передней и задней оси электротранспорта, а также оптимальные параметры емкости тяговых батарей \ldots
\end{enumerate}

%{\influence} \ldots

{\methods} 
\begin{enumerate}
	\item Теоретические основы электротехники
	\item Теория автоматического управления
	\item Математический анализ
	\item Методы численного моделирования (Simulink MATLAB)
\end{enumerate}
{\defpositions}
\begin{enumerate}
  \item Определение возможностей расширения диапазона рабочих скоростей электромотора в зависимости от алгоритма управления при подключении двум двухуровневым инверторам к началу и концу обмоток электромотора с компенсирующей емкостью
  \item Определение параметров номинальной мощности электромоторов, емкости тяговых батарей следует осуществлять в зависимости от нагруженности оси в статическом и динамическом режимах
  \item Двухинверторный привод с использованием аккумулятора и конденсаторной батареи в качестве источников энергии является наиболее энергоэффективной для электротранспорта
  %\item Четвертое положение
\end{enumerate}

{\reliability} полученных результатов обеспечивается \ldots \ Результаты находятся в соответствии с результатами, полученными другими авторами.

{\probation}
Основные результаты работы докладывались~на:
перечисление основных конференций, симпозиумов и~т.\:п.

{\contribution} Автор принимал активное участие \ldots

\ifnumequal{\value{bibliosel}}{0}
{%%% Встроенная реализация с загрузкой файла через движок bibtex8. (При желании, внутри можно использовать обычные ссылки, наподобие `\cite{vakbib1,vakbib2}`).
    {\publications} Основные результаты по теме диссертации изложены в XX печатных изданиях,
    X из которых изданы в журналах, рекомендованных ВАК,
    X "--- в тезисах докладов.
}% 
{%%% Реализация пакетом biblatex через движок biber
    \begin{refsection}[bl-authorvak,bl-authorwos,bl-authorscopus,bl-authorother,bl-authorconf]
        % Это refsection=1.
        % Процитированные здесь работы:
        %  * подсчитываются, для автоматического составления фразы "Основные результаты ..."
        %  * попадают в авторскую библиографию, при usefootcite==0 и стиле `\insertbiblioauthor` или `\insertbiblioauthorgrouped`
        %  * нумеруются там в зависимости от порядка команд `\printbibliography` в этом разделе. 
        %  * при использовании `\insertbiblioauthorgrouped`, порядок команд `\printbibliography` в нём должен быть тем же (см. biblio/biblatex.tex)
        %
        % Невидимый библиографический список для подсчёта количества публикаций:
        \printbibliography[heading=nobibheading, section=1, env=countauthorvak,    keyword=biblioauthorvak]%
        \printbibliography[heading=nobibheading, section=1, env=countauthorwos,    keyword=biblioauthorwos]%
        \printbibliography[heading=nobibheading, section=1, env=countauthorscopus, keyword=biblioauthorscopus]%
        \printbibliography[heading=nobibheading, section=1, env=countauthorconf,   keyword=biblioauthorconf]%
        \printbibliography[heading=nobibheading, section=1, env=countauthorother,  keyword=biblioauthorother]%
        \printbibliography[heading=nobibheading, section=1, env=countauthor,       keyword=biblioauthor]%
        %
        % Цитирования.
        %  * Порядок перечисления определяет порядок в библиографии (только внутри подраздела, если `\insertbiblioauthorgrouped`).
        %  * Если не соблюдать порядок "как для \printbibliography", нумерация в `\insertbiblioauthor` будет кривой.
        %  * Если цитировать каждый источник отдельной командой --- найти некоторые ошибки будет проще.
        %
        %% authorvak
        \nocite{vakbib1}%
        \nocite{vakbib2}%
        %
        %% authorwos
        %\nocite{wosbib1}%
        %
        %% authorscopus
        \nocite{scbib1}%
        %
        %% authorconf
        \nocite{confbib1}%
        \nocite{confbib2}%
        %
        %% authorother
        \nocite{bib1}%
        \nocite{bib2}%
        %
        %
        {\publications} Основные результаты по теме диссертации изложены в~\arabic{citeauthor}~печатных изданиях,
        \newcounter{citeauthorscwostot}% сумма citeauthorscopus и citeauthorwos
        \setcounter{citeauthorscwostot}{\value{citeauthorscopus}}%
        \addtocounter{citeauthorscwostot}{\value{citeauthorwos}}%
        \arabic{citeauthorvak} из которых изданы в журналах, рекомендованных ВАК\sloppy%
        \ifnum \value{citeauthorscwostot}>0%
            , \arabic{citeauthorscwostot} "--- в~периодических научных журналах, индексируемых Web of Science и Scopus\sloppy%
        \fi%
        \ifnum \value{citeauthorconf}>0%
            , \arabic{citeauthorconf} "--- в~тезисах докладов.
        \else%
            .
        \fi
    \end{refsection}%
    \begin{refsection}[bl-authorvak,bl-authorwos,bl-authorscopus,bl-authorother,bl-authorconf]
        % Это refsection=2.
        % Процитированные здесь работы:
        %  * попадают в авторскую библиографию, при usefootcite==0 и стиле `\insertbiblioauthorimportant`.
        %  * ни на что не влияют в противном случае
        \nocite{vakbib2}%vak
        \nocite{bib1}%other
        \nocite{confbib1}%conf
    \end{refsection}%
	%
	
	% Всё, что вне этих двух refsection, это refsection=0,
	%  * для диссертации - это нормальные ссылки, попадающие в обычную библиографию
	%  * для автореферата:
	%     * при usefootcite==0, ссылка корректно сработает только для источника из `external.bib`. Для своих работ --- напечатает "[0]" (и даже Warning не вылезет).
	%     * при usefootcite==1, ссылка сработает нормально. В авторской библиографии будут только процитированные в refsection=0 работы.
}

При использовании пакета \verb!biblatex! для автоматического подсчёта
количества публикаций автора по теме диссертации, необходимо
их~перечислить с использованием команды \verb!\nocite! в \verb!common/characteristic.tex!
