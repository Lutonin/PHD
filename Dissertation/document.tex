
\section{Постановка задач диссертационной работы}
\label{sec:ch1/sec4}


Сошлёмся на библиографию.
Одна ссылка: \cite[с.~54]{Sokolov}\cite[с.~36]{Gaidaenko}.
Две ссылки: \cite{Sokolov,Gaidaenko}.
Много ссылок: %\cite[с.~54]{Lermontov,Management,Borozda} % такой «фокус»
%вызывает biblatex warning относительно опции sortcites, потому что неясно, к
%какому источнику относится уточнение о страницах, а bibtex об этой проблеме
%даже не предупреждает
\cite{Lermontov, Management, Borozda, Marketing, Constitution, FamilyCode,
Gost.7.0.53, Razumovski, Lagkueva, Pokrovski, Methodology, Nasirova, Berestova,
Kriger}%
\ifnumequal{\value{bibliosel}}{0}{% Примеры для bibtex8
    \cite{Sirotko, Lukina, Encyclopedia}%
}{% Примеры для biblatex через движок biber
    \cite{Sirotko2, Lukina2, Encyclopedia2}%
}%
.
И~ещё немного ссылок:
\cite{Article,Book,Booklet,Conference,Inbook,Incollection,Manual,Mastersthesis,
Misc,Phdthesis,Proceedings,Techreport,Unpublished}
% Следует обратить внимание, что пробел после запятой внутри \cite{}
% обрабатывается ожидаемо, а пробел перед запятой, может вызывать проблемы при
% обработке ссылок.
\cite{medvedev2006jelektronnye, CEAT:CEAT581, doi:10.1080/01932691.2010.513279,
Gosele1999161,Li2007StressAnalysis, Shoji199895, test:eisner-sample,
test:eisner-sample-shorted, AB_patent_Pomerantz_1968, iofis_patent1960}
\ifnumequal{\value{bibliosel}}{0}{% Примеры для bibtex8
}{% Примеры для biblatex через движок biber
    \cite{patent2h, patent3h, patent2}%
}%
.

\ifnumequal{\value{bibliosel}}{0}{% Примеры для bibtex8
Попытка реализовать несколько ссылок на конкретные страницы
для \texttt{bibtex} реализации библиографии:
[\citenum{Sokolov}, с.~54; \citenum{Gaidaenko}, с.~36].
}{% Примеры для biblatex через движок biber
Несколько источников (мультицитата):
% Тут специально написано по-разному тире, для демонстрации, что
% применение специальных тире в настоящий момент в biblatex приводит к непоказу
% "с.".
\cites[vii--x, 5, 7]{Sokolov}[v"--~x, 25, 526]{Gaidaenko}[vii--x, 5, 7]{Techreport},
работает только в \texttt{biblatex} реализации библиографии.
}%

Ссылки на собственные работы:~\cite{vakbib1, confbib1}

Сошлёмся на приложения: Приложение \ref{app:A}, Приложение \ref{app:B2}.

Сошлёмся на формулу: формула \eqref{eq:equation1}.

Сошлёмся на изображение: рисунок \ref{fig:knuth}.

Стандартной практикой является добавление к ссылкам префикса, характеризующего тип элемента.
Это не является строгим требованием, но позволяет лучше ориентироваться в документах большого размера.
Например, для ссылок на рисунки используется префикс \textit{fig},
для ссылки на таблицу -- \textit{tab}.

В таблице \ref{tab:tab_pref} приложения \ref{app:B4} приведён список рекомендуемых
к использованию стандартных префиксов.

\section{Формулы} \label{sec:ch1/sec0}